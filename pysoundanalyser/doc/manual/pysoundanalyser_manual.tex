\documentclass[a4paper,12pt]{report}
\usepackage[pdftex]{graphicx}

%\usepackage[hmargin={3.5cm}]{geometry}
\usepackage{mathpazo,bm}
%\usepackage[T1]{fontenc}
%\usepackage{lmodern} %lmodern,euler
\usepackage{afterpage}
\usepackage{booktabs}
\usepackage{longtable}
\usepackage{multicol}
\usepackage{fancyvrb}
%\usepackage{graphicx}
\usepackage{caption}
%\usepackage{keystroke}
\usepackage{makeidx}
%\usepackage{floatflt}
\usepackage{wrapfig}
\captionsetup[table]{aboveskip=5pt,belowskip=5pt,position=top,margin=10pt,font=small,labelfont=bf,format=hang}
\captionsetup[figure]{aboveskip=0pt,belowskip=0pt,position=top,margin=10pt,font=small,labelfont=bf,format=hang}

\makeindex

%\setlength{\parindent}{10pt}
\setcounter{secnumdepth}{2}

\usepackage[hyperindex, bookmarks=true, bookmarksnumbered=true, breaklinks=true]{hyperref}
\hypersetup{pdftitle=pysoundanalyser manual,
            pdfauthor=Samuele Carcagno,
            pdfsubject=Manual for pysoundanalyser,
            pdfpagemode=UseOutlines,
            pdfstartview=FitH,
            pdfkeywords={pysoundanalyser, python, sound, analyser},
            colorlinks=true,
            linkcolor=black,
            citecolor=black,
            filecolor=black,
            urlcolor=blue  
} 
%pdfpagemode=none #for no bookmarks


\begin{document}
\include{titlepage}
\reversemarginpar
\newcommand{\sq}{\texttt{"}}           %{\textquotedbl}
%\newcommand{\pycho}{\texttt{pychoacoustics}}
\renewcommand{\bibname}{References}

\newcommand*{\main}[1]{\textbf{\hyperpage{#1}}}

%\include{titlepage}
\pagenumbering{gobble}


\pagenumbering{roman}
\tableofcontents

\clearpage
\pagenumbering{arabic}

\chapter{Introduction}



%%% Local Variables: 
%%% mode: latex
%%% TeX-master: "pysoundanalyser_manual"
%%% End: 


\include{user_interface}

% \appendix
% \renewcommand{\arraystretch}{1.5}
% \include{}


% \cleardoublepage
% \phantomsection
% \addcontentsline{toc}{chapter}{References}
% \bibliographystyle{abbrv}
% \bibliography{r}
% \nocite{*}

% \cleardoublepage
% \phantomsection
% \addcontentsline{toc}{chapter}{Index}
% \printindex

\end{document}